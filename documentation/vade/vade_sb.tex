%**********************************************************************

\documentclass[a5paper, oneside, 10pt, titlepage, noindent, just=true]{scrbook}

\usepackage[latin1]{inputenc}

\usepackage[osf, sc]{mathpazo}

\usepackage{graphicx}

\usepackage{makeidx}

\usepackage{mathtools}

\usepackage{alltt}

\renewcommand{\encodingdefault}{T1}

%usepackage[ocr-a]{ocr}
%usepackage[T1]{fontenc}
%renewcommand{\ttdefault}{ocr}

\setkomafont{disposition}{%
	\normalfont}

%*******************************************
% use the following commands all or nothing:
%usepackage[T1]{fontenc}
%renewcommand*\familydefault{\ttdefault}

%usepackage[just=true]{stdpage}
%*******************************************

%**********************************************************************
% additional commands for LaTeX/the Yabasic Vademecum
%**********************************************************************

\newcommand{\SB}{\textsc{SmallBASIC}}

\newcommand{\SBversion}{V 0.10.16}

\newcommand{\blindtext}{Lorem ipsum dolor sit amet}

%**********************************************************************

%\newcommand{\newFeature}{\marginpar{\textsf{\small \begin{center}
%\includegraphics[width=0.8cm]{icon/new} \\ New in V3\end{center}}}}

% allows the use of funky icons on the right margin for warnings etc.

%**********************************************************************

% what does this comment do?
\newsavebox{\fmbox}
\newenvironment{fmpage}[1]
{\begin{lrbox}{\fmbox}\begin{minipage}{#1}}
{\end{minipage}\end{lrbox}\fbox{\usebox{\fmbox}}}

% give a quick summary of the chapter:

%newcommand{\quick}[1]{\begin{itemize} \item #1 \end{itemize}}

\newcommand{\quick}[1]{\fbox{\parbox{\textwidth}{#1}}}

\newcommand{\gist}[1]{\marginpar{\footnotesize #1}}

% display something as code:
\newcommand{\co}[1]{\texttt{{#1}}}

\newcommand{\Co}[1]{>>\co{#1}<<}

% references to commands start with "c:"
% references to sections start with "c"

% insert a reference to a different command in the manual section:
\newcommand{\sbref}[1]{\hyperref[c:#1]{\co{#1}}}

% insert a reference to a chapter. Argument is a label.
\newcommand{\Cref}[1]{\ref{#1}\ \nameref{#1}}

% create a complete entry in the manual section:
\newcommand{\vCommand}[4]{\subsection*{\co{#1} \label{c:#1}} {#2} \par
\medskip \textbf{Syntax:} \\ \co{#3} \par \medskip \textbf{Description:}
\\ {#4} \par \medskip}
%	* command name
%	* short description
%	* syntax
%	* long description

\newcommand{\vCommend}[1]{\par \textbf{See also: #1}}

\newcommand{\at}{\makeatletter @\makeatother}

%**********************************************************************

\newcommand{\stringCausesError}{Passing a string value or a string
expression raises a syntax error.}

\newcommand{\numericCausesError}{Passing a numerical constant, variable
or expression raises a syntax error.}

\newcommand{\twoEquiv}{The following two code segments are equivalent:}


%**********************************************************************
\newcommand{\vquote}[2]{\begin{quote} \raggedleft \emph{#1} \\ {#2}
\end{quote}}


\usepackage{setspace}

%onehalfspacing

\usepackage{microtype}

\lefthyphenmin=3

\usepackage{array}

\usepackage{color}

\definecolor{lightgray}{gray}{0.9}

\usepackage{listings}

\lstset{ %
language=,                 % choose the language of the code
basicstyle=\ttfamily \footnotesize,       % the size of the fonts that are used for the code
numbers=none,                   % where to put the line-numbers
numberstyle=\footnotesize,      % the size of the fonts that are used for the line-numbers
columns=fullflexible            % better kerning?
stepnumber=2,                   % the step between two line-numbers. If it's 1 each line will be numbered
numbersep=5pt,                  % how far the line-numbers are from the code
backgroundcolor=\color{lightgray},  % choose the background color. You must add \usepackage{color}
showspaces=false,               % show spaces adding particular underscores
showstringspaces=false,         % underline spaces within strings
showtabs=false,                 % show tabs within strings adding particular underscores
frame=none,	                % adds a frame around the code  none or single
tabsize=4,	                % sets default tabsize to 2 spaces
captionpos=b,                   % sets the caption-position to bottom
breaklines=true,                % sets automatic line breaking
breakatwhitespace=false,        % sets if automatic breaks should only happen at whitespace
title=\lstname,                 % show the filename of files included with \lstinputlisting; also try caption instead of title
aboveskip=\medskipamount,
belowskip=0pt,
escapeinside={\%*}{*)}          % if you want to add a comment within your code
}

\usepackage{hyperref}

\hypersetup{
    colorlinks=true,       % false: boxed links; true: colored links
    linkcolor=blue,          % color of internal links
    citecolor=green,        % color of links to bibliography
    filecolor=magenta,      % color of file links
    urlcolor=cyan           % color of external links
}

\setcounter{secnumdepth}{4}

\setcounter{tocdepth}{2}

% prettify the design of \paragraphs:
\makeatletter
\renewcommand\paragraph{%
   \@startsection{paragraph}{4}{0mm}%
      {-\baselineskip}%
      {.5\baselineskip}%
      {\normalfont\normalsize}}
\makeatother

\setcounter{secnumdepth}{3}		% we don't want to number paragraphs

\hyphenation{small-basic}

\makeindex

%**********************************************************************

\newcommand{\absatz}{\bigskip}

%**********************************************************************

\frontmatter

\title{The \\ \SB\ >>Vademecum<< \\ (\SBversion) \\\emph{Incomplete --
Draft}}

\author{Elmar Vogt \\ F�rth, GERMANY}

%**********************************************************************

\begin{document}


\newcommand{\qA}{\vquote{>>The process of preparing programs for a digital computer is
especially attractive because it not only can be economically and
scientifially rewarding, it can also be an aesthetic experience much
like composing poetry or music.<<} {-- \textsc{Donald F. Knuth}}}

\newcommand{\qB}{\vquote{>>Code: A set of symbol whose primary purpose is to restrict
comprehension.<<} {-- Webster's 3rd International Dictionary}}

\newcommand{\qC}{\vquote{>>The most amazing achievement of the computer software industry
is its continuing cancellation of the steady and staggering gains made
by the computer hardware industry.<<} {-- \textsc{Henry Petroski}}}

\newcommand{\qD}{\vquote{>>It has been said that the great scientific disciplines are
examples of giants standing on the shoulders of other giants.  It has
also been said that the software industry is an example of midgets
standing on the toes of other midgets.<<} {-- \textsc{Alan Cooper}}}

\newcommand{\qE}{\vquote{>>That�s the thing about people who think they hate computers. 
What they really hate is lousy programmers.<<} {-- \textsc{Larry
Niven}}}

\newcommand{\qF}{\vquote{>>The Hitchhiker's Guide to the Galaxy \ldots\ says of the
Sirius Cybernetics Corporation products that "it is very easy to be
blinded to the essential uselessness of them by the sense of achievement
you get from getting them to work at all.<<} {-- \textsc{Douglas Adams},
>>Life, the Universe, and Everything<<}}

\newcommand{\qG}{\vquote{>>Measuring programming progress by lines of code is like
measuring aircraft building progress by weight.<<} {-- \textsc{Bill
Gates}}}

\newcommand{\qH}{\vquote{>>Should array indices start at 0 or 1?  My compromise of 0.5
was rejected without, I thought, proper consideration.<<} {--
\textsc{Stan Kelly-Bootle}}}

\newcommand{\qI}{\vquote{>>Saying that Java is nice because it works on all OSes is like
saying that anal sex is nice because it works on all genders.<<} {--
\textsc{Alanna}}}

\newcommand{\qJ}{\vquote{>>Always code as if the guy who ends up maintaining your code
will be a violent psychopath who knows where you live.<<} {--
\textsc{Martin Golding}}}

\newcommand{\qK}{\vquote{>>On two occasions I have been asked, � >Pray, Mr. Babbage, if
you put into the machine wrong figures, will the right answers come
out?< In one case a member of the Upper, and in the other a member of
the Lower House put this question. I am not able rightly to apprehend
the kind of confusion of ideas that could provoke such a question.<<}
{-- \textsc{Charles Babbage}, >>Passages from the Life of a
Philosopher<< (1864), ch. 5: >>Difference Engine No. 1<<}}

\newcommand{\qL}{\vquote{>>Even perfect program verification can only establish that a
program meets its specification. \ldots\ Much of the essence of building
a program is in fact the debugging of the specification.<<} {--
\textsc{Fred Brooks}, >>No Silver Bullet<<, in >>The Mythical Man-Month
Anniversary Edition<< 1995}}

\newcommand{\qM}{\vquote{>>For twenty years programming languages have been steadily
progressing toward their present condition of obesity; as a result, the
study and invention of programming languages has lost much of its
excitement. Instead, it is now the province of those who prefer to work
with thick compendia of details rather than wrestle with new ideas.
Discussions about programming languages often resemble medieval debates
about the number of angels that can dance on the head of a pin instead
of exciting contests between fundamentally differing concepts.<<} {--
\textsc{John Backus}, >>Can Programming Be Liberated From the von
Neumann Style?"<<, 1977 Turing Award Lecture, Communications of the ACM
21 (8), (August 1978)}}

\newcommand{\qN}{\vquote{>>Documentation is like sex: when it is good, it is very, very
good; and when it is bad, it is better than nothing.<<}{-- \textsc{Dick
Brandon} (?)}}

\newcommand{\qO}{vquote{>>Voluminous documentation is part of the problem, not part of the
solution.<<}{-- \textsc{-- Tom DeMarco}, software development
consultant}}

%sloppy

\maketitle

\chapter{Foreword}

\qN

A \emph{Vademecum} (lat. >>vade-me-cum<<, >>Come with me<<) is a small
booklet on a subject, meant to be your handy companion. While not as
exhaustive as a genuine reference manual, it is designed to let you
quickly find information on a certain subject: This Vademecum omits some
details and deprecated features of \SB\ for the sake of brevity, as it
aims to help you find your way around this elegant language in a
friendly, trustworthy, and straightforward manner.

\absatz

The first part of the Vademecum will give you an introduction to the
language, while the second part can be used as a reference for work.

\absatz

-- Elmar Vogt

\tableofcontents

\mainmatter

%**********************************************************************

\part{Language Overview}

\chapter{A Quick and Painless Introduction}

\quick{This chapter is designed to give you an overview over \SB\ and
lets you determine whether the language suits your needs.}

\section{Two Words of Caution}

\begin{itemize}

\item This booklet is about the \SB\ programming language -- it's
\emph{not} an introduction to programming in general. While not a lot of
software background is expected from the reader, it is assumed that
common programming concepts are already understood.\footnote{If you're
struggling with terms like >>strings<<, >>loops<<, or >>pointers<<, this
booklet might not be for you.} Some knowledge in programming languages
like >>C<<, Pascal, or Ruby is useful.

\item \SB\ is a language with a long and varied history.  Some features
which are still present in its code are no longer actively supported.
These deprecated features may be removed in future releases, and, to
keep the Vademecum concise, they will not be dealt with in this
document.

\end{itemize}

\section{Features, or: Is it for me?}

\SB\ started life as something like an >>extended handheld calculator<<,
designed for PIMs\footnote{>>Personal Information Manager<< -- for you
youngsters out there, that's a smartphone without connectivity} running
the Palm OS. One release note read, >>It's not meant to be a
full-fledged programming language, and it will never be. Please don't
ask us to turn it into one.<<

Some time has gone by since, and \emph{Nicholas Christopoulos} and
\emph{Chris Warren-Smith}, the driving forces behind the project, have
developed \SB\ into a dialect of the BASIC language which is neither
>>small<< (in the sense of it's capabilities), nor does it share too
much with classic BASIC dialects.\footnote{On a completely unrelated
tangent, I'm convinced that one of the reasons BASIC has become much
less popular today than many of the more strictly standardized languages
like >>C<<, Python, or Ruby, is that though there
actually \emph{is} a standard for the language, it never has really been
implemented. Rather, every BASIC dialect shows its own strengths and
shortcomings, and one never really knows what one gets when one toys
with a new dialect. This is one of the charms of working with BASIC, but
of course it makes maintaining or porting software written in that
language a nightmare.\\Back to our scheduled programme.}  

Today, some of \SB's features are:

\begin{itemize}

\item \SB\ is a multi-platform BASIC language: Currently, Linux, 
Windows and Android are supported.\footnote{Older ports for Palm OS, DOS,
and several others are no longer actively maintained. While some of
these versions are still around, they may miss many of the features
described here.}

\item The language is pretty compact: The Debian installer for Linux,
for example, comes as a single file with ca. 600~kb.

\item \SB\ features a very comprehensive set of mathematical functions.

\item It is an interpreted language with no compilation runs required.

\item \SB\ supports structured programming, user-defined structures and
modularized source files.  It is not object-oriented, though.

\item It also shows much leeway in questions of syntax: For many
commands, there are alternatives, and for many constructs, there are
different synonyms available.

\item \SB\ comes with its own little IDE.

\item Graphics primitives (like lines, circles, etc.) are provided, as
well as sound and simple GUI functions.

\item \SB\ supports HTML and the Web in a rudimentary way.

\end{itemize}

\section{Resources \label{resources}}

Here you will find a few internet resources that might be helpful for you when
you want to get more closely acquainted with \SB:

\begin{itemize}

\item \url{http://smallbasic.sourceforge.net/} is the central hub for
information about \SB\ in general -- a good starting point for a user of
\SB. It leads you to the download of the current \SB\ versions and
provides a lot of background information.

\item \url{http://sourceforge.net/projects/smallbasic/} hosts the
source code and cutting-edge \SB\ releases -- most interesting if you want
to contribute to the further development of \SB.

\item There is also a small forum on the sourceforge site:
\url{http://smallbasic.sourceforge.net/?q=forum} It doesn't carry very
much traffic, but is a good point to ask questions about \SB\ or make
suggestions.

\item \url{https://www.facebook.com/groups/12117250426/} and
\url{https://www.facebook.com/pages/SmallBASIC/110997952286349} are two
Facebook groups dedicated to \SB. (Both don't exactly swamp you with
traffic either.)

\item \url{http://forum.basicprogramm.org} is a generally good place to
ask questions around various BASIC programming languages.

\item You can get directly in touch with the developing team of \SB\
through \href{mailto:smallbasic@gmail.com}{e-mail}.

\end{itemize}

\section{Licenses}

\begin{itemize}

\item{\SB\ is released under the
\href{http://www.gnu.org/licenses/old-licenses/gpl-2.0}{GNU General Public
License version 2.0 (GPLv2)}}

\item{This document, the >>Vademecum<<, is released under the
\href{http://creativecommons.org/licenses/by-nc-nd/3.0/de/deed.en_GB}{Creative
Commons License by-nc-nd}. In short, this license means that you are
free to reproduce and distribute this document in any non-commercial
manner, as long as you don't change the author's name (that's mine), and
as long as the contents remain unchanged.}

\end{itemize}



\chapter{Installation}

\quick{In this chapter you'll learn how to install \SB\ and prepare a
working environment with it. (There's really not much to it.)}

\absatz

You'll find the web address of the current software versions in the chapter
\Cref{resources}.

\section{Windows}

Go to the \href{http://smallbasic.sourceforge.net/?q=node/195}{Download
section} at \SB's Sourceforge site, and download the current Windows
version of \SB.

Run the executable, and you're set: You should by now see a \SB\ icon in
the start menu of Windows.

You can run your \SB\ programs in one of two ways:

\begin{itemize}

\item You can open the \SB\
IDE (see \Cref{ide}), load your program and run it from the IDE.

\item Alternatively, connect the filetype extension \Co{.bas} with \SB. From
then on a double-click on a \Co{.bas} file in the Windows file explorer
will launch the interpreter with your program.

\end{itemize}

\section{Linux \label{installLinux}}

Go to the \href{http://smallbasic.sourceforge.net/?q=node/195}{Download
section} at \SB's Sourceforge site, and download the appropriate Linux 
version of \SB:

\begin{itemize}

\item If you're running Ubuntu, download the Ubuntu version,

\item In all other cases, download the openSUSE version.

\end{itemize}

A double-click on the package will bring up your packet manager. From
there, follow the installation instructions. After a successful
installation, \SB\ will appear in Linux' >>Development<< start menu.

You can run \SB\ programs in three different ways:

\begin{itemize}

\item Start the IDE from the start menu, edit your file and run it from
there,

\item Open a terminal, and use the command line \Co{sbasici -r
filename.bas}, or

\item Double-click on a \SB\ file in the file explorer, and chose the
option >>Run<< in the subsequent dialog.

\end{itemize}

For the last option, the very first line in your basic file must be

\begin{lstlisting}
#! /usr/bin sbasici
\end{lstlisting}

(This mechanism of linking a source file to a hosting program is called
>>Shebanging<< in Unix-like OS's. Read the
\href{http://en.wikipedia.org/wiki/Shebang_(Unix)}{Wikipedia
article} to learn more about it.) \index{\# (>>shebang<<)}

\section{Android}

Go to 
\href{https://play.google.com/store/apps/details?id=net.sourceforge.smallbasic}{\SB's
Google play store entry}, and click the >>Install<< button.

When starting \SB\ on your smartphone, you will see a rudimentary IDE
to launch your programs. (See \Cref{ide_android})


\chapter{The Integrated Development Environment (IDE) \label{ide}}

I don't use it.

\section{Windows and Linux}

\section{Android \label{ide_android}}


chapter{The Basics of a Program: \SB\ Fundamentals}

\quick{Here you learn about the basics of BASIC, so to speak. This
chapter deals with the way a \SB\ program composed.}

\section{Source Code Format}

A program consists of a single source file. It is possible to include
libraries from external source files with the Unit-mechanism (see
\Cref{modules}).

Source files are simple text files. They can be
written in \textbf{ASCII} or \textbf{UTF-8}.

The basic program component is a line of text, ending with \Co{CR} or
\Co{LF/CR} characters.\footnote{Don't worry about this, your operating
system will handle it right. It may only every be an issue if you use
source files written in one OS and then transferred to a different one.}

\SB\ is \textbf{case-insensitive}: The names \Co{myvar}
and \Co{MyVar} will always refer to the same variable or function.
Likewise, keywords are case-insensitive: Both \Co{print} and \Co{PRINT}
are legal variants of the same command. \index{case sensitivity}

\textbf{Whitespace} -- i.e., non-printing characters like
spaces and tabs -- is ignored in \SB, except inside string literals
where it is retained (see \Cref{stringLiterals}). \twoEquiv\footnote{All listings
in the vademecum follow the same convention and show the source code as
you would have typed it in in the IDE. If a line in the listing begins
with a greater-than sign \Co{>}, this indicates a response of the
program printed on your screen.} \index{whitespace} 

\begin{lstlisting}
for a = 0 to 10

for a=0to 10
\end{lstlisting}

But note that the ommission of whitespace can lead to parsing errors: If
the above line were abbreviated to

\begin{lstlisting}
fora=0to10
\end{lstlisting}

this would cause an error, because \Co{for} and \Co{to} wouldn't be
recognized as keywords anymore. Rather, \SB\ would consider \Co{fora}
and \Co{to10} to be variable names.\footnote{As a rule of thumb, it's
advisable to always leave spaces around keywords.}

Each program line contains one or more commands.  Multiple commands on a
line are \textbf{seperated by a colon} \Co{:}. \index{: (command
seperator)} \twoEquiv

\begin{lstlisting}
print "Hello world!"
print "Wonderful day."

print "Hello world!" : print "Wonderful day."
\end{lstlisting}

\textbf{Line continuation}: If the ampersand \Co{\&} is the last character on a line, then the
interpreter will assume that the current command extends to the next
line as well. \index{\& (line continuation char.)} \twoEquiv

\begin{lstlisting}
X = 245 * 198 - sqr(5)

X = 245 * 198 &
		- sqr(5)
\end{lstlisting}

\section{Literals}

\subsection{Numbers}

Numbers can be written in the usual manner, using either
>>conventional<< or scientific notation. All of the following examples
are legal numbers in \SB:

\begin{lstlisting}
1, 0, -1, 1.2, -23232.5, 1.902e-50, -.423
\end{lstlisting}

As is shown in the last example, numbers with an absolute value $<1$
need not be preceded with \Co{0}.

Integer numbers can also be represented in hexadecimal, octal and binary
notation with various prefixes:

\begin{description}

\item[hexadecimal]: \Co{\&h}, \Co{0h}, \Co{\&x}, or \Co{0x}

\item[octal]: \Co{\&o} or \Co{0o})

\item[binary]: \Co{\&b} (or \Co{0b})

\end{description}

\index{\&h (hexadec. number prefix)} \index{\&o (octal number prefix)}
\index{\&b (binary number prefix)} \index{$0$h (hexadec. number prefix)}
\index{$0$o (octal number prefix)} \index{$0$b (binary number prefix)}
\index{$0$x (hexadec. number prefix)} \index{\&x (hexadec. number prefix)} 

\begin{lstlisting}
&hAFFE0815, &o4242, &b100101011
0hAFFE0815, 0o4242, 0b100101011
\end{lstlisting}

/* limits for numbers? */

\subsection{String literals \label{stringLiterals}}

String literals are character sequences which are to be treated as
program data >>as is<<, not as variable or keyword names. String
literals are bracketed by double quotes \Co{"}.  \index{string literals}
\index{'' (string delimiter)}

\begin{lstlisting}
"This is a string literal"
this will be considered as a sequence of keywords
\end{lstlisting}

Note that when a string literal is to be extended across more than one
line, it must be properly closed before the continuation ampersand and
re-opened on the subsequent line with the \Co{"} delimiter character:

\begin{lstlisting}
print "Hello &
		world!"		' error

print "Hello " &
		"world!"		' correct
\end{lstlisting}

\section{Identifiers}

Identifiers -- >>names<< for variables and functions -- follow the usual
conventions: \index{variable names} \index{function names} \index{procedure
names}

They consist of a letter or an underscore \Co{\_}, followed by one or
more of the following:\begin{itemize}

\item other letters

\item digits (\Co{0} -- \Co{9})

\item the underscore \Co{\_} \index{\_ (underscore)}

\item the dollar sign \Co{\$} (only as the last character of the
identifier, see below) \index{\$ (dollar sign)}

\end{itemize}

A single underscore \Co{\_} is a legal complete identifier.

Identifiers can have virtually unlimited length. All characters are
significant in resolving an identifier (ie, to determine whether two
identifiers refer to the same variable.)\footnote{This is in contrast to
many older BASICs. The \emph{Commodore BASIC} shipped with the honorable
C64, for example, allowed identifiers of arbitrary length, but used only
the first two letters for resolution: \Co{hoogla} and \Co{hooray} were
considered to refer to the same variable.}

Traditionally, in BASIC the \textbf{dollar sign} \Co{\$} serves as a
sigil to indicate that a name identifies a string variable, if used as
the last character of the identifier (i.e., \Co{my\_name\$}).

Since \SB\ is a typeless language (see below) where variables can hold
values of any type, such a sigil would be misleading, yet it has been
retained for the sake of compatibility. It may be placed as the last
character of an identifier only. Here it serves two distinguish between 
identifiers (\Co{harry} and \Co{harry\$} are two different identifiers),
but has otherwise no function.

\section{Comments}

\textbf{Line comments} can be introduced in three ways:

\begin{itemize}

\item With the keyword \Co{rem},

\item With the apostrophe character \Co{'}, \index{' (comment
introduction)}

\item With a hash sign \Co{\#}. \index{\# (comment introduction)}

\end{itemize}

Everything on the current line following the comment introduction will
be ignored in program execution.

If the \Co{rem} keyword is used and it is preceded by other commands on
the current line, it must be seperated from the previous commands by a
colon \Co{:} If the hash sign is used, it must be the first character on
the line. (See also the use of a hash sign in >>shebanging<< a script,
\Cref{installLinux})

\begin{lstlisting}
for a=0 to 10      ' this is a valid comment
   print a         : rem this also
# this is a whole line commented out
next a
rem the last comment

print "Hello world!"   rem vain commenting attempt
\end{lstlisting}

The \Co{rem} in the last line above will cause an error, because it
needs to be preceded with a colon \Co{:}.

There are no provision for \textbf{block comments}.




\chapter{Data}

\quick{In this chapter, the Vademecum introduces you to the various ways
\SB\ handles program data -- both simple variables and more complex
structures.}

\section{A Note on Types \label{noteOnTypes}}

\SB\ is \textbf{dynamically typed language}\footnote{sometimes also
referred to somehow incorrectly as a >>typeless<< language}. This means
that one and the same variable can hold numerical values and strings at
different times. \index{type system} Furthermore, necessary conversions
are done automatically on the fly, for example when a string is used as
the parameter for a numerical function:

\begin{lstlisting}
x= "3.141"    ' string assignment
y= cos(x)     ' no problem:
' x is automatically converted to a number
print y

> -1
\end{lstlisting}

While there are also functions for explicit conversions, per default
conversions are done >>tacitly<< by the interpreter, without any
provisions in the \SB\ program, and there are also no >>notifications<<
when a type conversion occurs. Another consequence is that the kind of
data contained in a variable (or the structure of a map, see \Cref{map})
is only determined at runtime.\footnote{Contrast this behaviour to other
languages like >>C<<, where a strict type discipline is enforced.} It is
up to the programmer to ensure that his code will >>expect the
unexpected<< and be able to cope with any data it is fed with.

\section{Simple Variables}

Simple variables \textbf{need not be declared}; they come into existance by their
first appearance in the source code. If they are not created through an
assignment, they will be initiated to the value >>$0$<< (or,
equivalently, to the empty string >>""<<. (This example admittedly looks a
little silly.) \index{variable declaration}

\textbf{Value assignment} is done with the \Co{$=$} operator: \index{=
(assigment operator)} The value
to the right of it (which may be another variable, a literal, or a
more complicated expression) is assigned to the variable on the
left:\footnote{The command \Co{?} used below is a shorthand for
\Co{print}: It will display the current values of all the following literals or
variables on the screen.}

\begin{lstlisting}
x= 20
y= 10
z= x+y
? z

>30

a= "Hello world"
? a

Hello world

"Goodbye world"= q
\end{lstlisting}

The last line will cause an error, since the operands are in the wrong
order: It's not possible to assign a variable like \Co{q} to a literal,
only the other way around.

Historically, BASIC required the keyword \Co{let} before an assignment:

\begin{lstlisting}
let x=20
\end{lstlisting}

\SB\ offers you the option to use this syntax variant for compatibility
reasons (read: nostalgia), but it's deprecated.

\subsection{Numbers}

Many languages have different types for various flavours of numbers --
signed or unsigned integers, reals, etc. --~ In contrast, \SB\ only has
a single class of numbers.\footnote{While \SB\ internally also
represents numbers either as integers or, if they carry fractions or
exceed the limits for integers, as 64 bit reals, this is invisible to
the user, since all conversions are done implicitly and automatically
when required.}

\SB\ numbers can have absolute values roughly between $2.2\cdot
10^{-208}$ (smaller values are considered $0$) and $1.8\cdot 10^{308}$
(larger values raise a flag \Co{INF} (for >>infinity<<).

\subsection{Strings}

Strings are chains of one or several letters, used to represent words,
sentences or complete texts. A string consisting of zero letters is
called an >>empty<< string. \index{empty string} (While I have a strong
hunch that strings are internally represented with UTF-8 unicode
characters, it's probably safer to only assign ASCII letters to them.)

\begin{lstlisting}
my_name= "Elmar Vogt"
? "Hello, my name is ", my_name

> Hello, my name is Elmar Vogt
\end{lstlisting}

Strings are dynamic objects in \SB, meaning that they have no
predefined length: In the course of a program, they can change their
length arbitrarily, and there is no need to define a maximum length for
them. A theoretical upper length limit is set at 2 billion
characters.\footnote{Even before that \SB\ tends to get too tediously
slow for all practical purposes.}

\textbf{String concatenation} is done with the $+$ operator: \index{+
(string concatenation operator)}

\begin{lstlisting}
my_name= "Elmar Vogt"
greeting= "Hello, my name is "
welcome= greeting + name
? welcome

> Hello, my name is Elmar Vogt
\end{lstlisting}

The $+$ operator combines the string operands to the left and to the
right of it to a new, third string, the length of which is the sum of
the individual strings' lengths. If either the left or the right operand
is a number -- a literal or a variable --, then this operand will be
converted to a string before the concatenation. For example, a variable
with the value \Co{42} would be converted to a string of two characters,
namely \Co{4} and \Co{2}.\footnote{Obviously, if \emph{both} operands are
numbers, $+$ will perform a simple addition and assign the sum of both
values to the variable on the left of the equality sign.}

The \textbf{length} of a string \index{length (string)} is simply the
number of characters currently contained in the string:\footnote{The
function \Co{len()} returns the length of the following argument in
brackets}

\begin{lstlisting}
? len("Hello world")

> 11
\end{lstlisting}

\textbf{String indexing} \index{indexing (strings)} is done with the first
character of a string being considere to be at position $1$:\footnote{The
function \Co{mid(x, y, z)} returns $z$ characters from the string $x$,
beginning with the $y$th}

\begin{lstlisting}
? mid("Help me", 4, 1)

> p
\end{lstlisting}

This also means that the index of the \emph{last} character of a string
is equivalent to its length:\footnote{The function \Co{len()} returns
the length of its string argument, in number of characters}

\begin{lstlisting}
z= "Help me"
? mid(z, len(z), 1)

> e
\end{lstlisting}

A loop designed to iterate over all characters in a string \Co{x} must thus
look like this:

\begin{lstlisting}
for i=1 to len(x)
	...
next i
\end{lstlisting}

Note that this in contrast with the use of arrays (see below), where the first
array element has the index $0$, and for an array with $n$ elements the
highest index is $n-1$.

\section{Complex data structures}

Beside simple variables and literals, \SB\ also offers the option of
composing arbitrarily complex\footnote{>>Complex<< here denoting
intricate constructs, not the mathematical notion of complex numbers}
variable structures. These come in two flavours: 

Simple arrays (\Cref{array}), and maps (\Cref{map}).

From \SB's point of view, both are the same. Technically, arrays are
simply a subset of possible maps, but to make it easier to grasp the
concept, we'll treat both as different entities for the moment.

In contrast to simple variables, complex data structures \textbf{must be
declared before use} with the \Co{dim} statement.

\begin{lstlisting}
dim x
\end{lstlisting}

Since (almost) all variables are handled dynamically in \SB\ and can change their
structure during their lifetime, it is neither necessary nor useful
to define details or the size of the data structure at hand.

\subsection{Arrays \label{array}}

Arrays are the more simple way of agglomerating data into a single
variable. \SB\ treats them much like the way other programming languages
do. An array holds a number of variables which are accessed by means of
the array name, and a numerical index, pretty much like a street address
is a combination of the street's name, and a house number. To access an
array member, its name is followed by the index in parentheses \Co{()}:

\begin{lstlisting}
hoogla(i)= boogla(250)
\end{lstlisting}

will assign the value of \Co{boogla}'s 250th member to the member of
\Co{hoogla} with the numberical index \Co{i}.

Array members can be of \textbf{mixed content}, ie it's perfectly okay
for one member of an array to hold a number, and for a different member
of the same array to hold a string.

Arrays must be \textbf{defined before use}, and while they are dynamic
and can change their size on the fly, room for an array member must be
defined before it can be used, by using the \Co{dim} statement along
with a numerical value:

\begin{lstlisting}
dim hoogla(250)
\end{lstlisting}

will define \Co{hoogla} to initially have $251$ members with indices
$0..250$. This \textbf{array indexing} \index{indexing (arrays)} 
contrasts with the indexing of strings.

Array members can be erased from the array with \Co{delete}. Note that
if the $i$th member of an array is deleted, then all array members with
higher indices >>move down<< one notch, ie the value of \Co{x(n)} will
now be in \Co{x(n-1)}. \index{array member deletion}

Likewise, new array members can be appended to an array with the
operator \Co{$<<$}.\footnote{Sorry for the ugly typography here.}

\begin{lstlisting}
dim hoogla(250)
hoogla << 132
\end{lstlisting}

means that now \Co{hoogla} will have $252$ members, and the new $252$nd
member (with the index $251$) will have the value $132$. \index{array
member addition} \index{$<<$ (array appendix operator)}

Arrays are not limited to being a chain of values. \textbf{Several array
dimensions} can be defined to render arrays of >>rectangular<<, >>cubic<<,
>>tesseractic<< or even higher dimensions:

\begin{lstlisting}
dim hoogla(10, 20, 10)
hoogla(5, 9, 8)= "blubbdi"
\end{lstlisting}

will create an array with $11 \times 21 \times 11$ members, or access
one particular member, resp. For obvious reasons, it's not possible to
\Co{delete} members of such a higher-dimensional array in the
abovementioned way. Likewise, using the appendix operator \Co{$<<$} with
a higher-dimensional array should be avoided.\footnote{While technically
it works, it will actually transform the array into a
\emph{one-dimensional} array, arrange the previous values sequentially
and finally append the new array member. But this is dark voodoo and
should not be practiced. Besides, there is no guarantee that this
behaviour will be retained in future versions of \SB.}

The maximum array size is virtually unlimited. But note that with the
introduction of new array dimensions, the space and performance
requirements increase exponentially.

\subsection{Maps \label{map}}

Maps differ from arrays in two ways: Firstly, while arrays always have a
linear or >>rectangular<< structure, \textbf{maps are >>data trees<<, where each
map member can be a simple variable, an array or a complete map}. This
structure can become extremely complex during the runtime of a program.
As explained in \Cref{noteOnTypes}, since there is no fixed type system,
and hence no predetermined structure for any given map, this means that
an \SB\ program must >>anticipate all known unknowns<<.

Secondly, map indices aren't limited to a consecutive list of integer
numbers. Rather, \textbf{map members can be accessed by any simple variable --
string, integer or real}. This leads to subtle syntax differences when
accessing them and has a number of other consequences. For example, it's
not straightforward anymore to write a loop which will iterate over all
map members, or determine the number of map elements. The concept is
probably more easy to grasp when one doesn't think of maps as of
traditional arrays, but to consider each member a pair of a >>value<<,
which is stored in the map, and a >>key<< (the index) by which it is
accessed.

These two features in combination with \SB's automatic conversion features
lead to the fact that \textbf{it is fairly easy to inadvertently convert an
array into a map}, a conversion from which there is no easy way back, and
thus to create havoc at runtime. 

\textbf{Map initialisation} can be done by in three ways: Either
explicitly, by using the \Co{dim} keyword \emph{without} an array size
(ie, simply \Co{dim x} is enough), through the \Co{array} keyword (see
below), or implicitly by assigning the map members values by a sequence
of comma-seperated values, enclosed in square brackets \Co{[]}:
\index{[] (map initialiser)} \index{map initialisation}

\begin{lstlisting}
hoogla= [10, 9, 8, 7, 6, 1]
\end{lstlisting}

initialises a simple map \Co{hoogla} with $6$ members with automatically
generated indices from $0$ to $5$. 

To create more complex structures, each map member which is a map again
must be enclosed in brackets. For example,

\begin{lstlisting}
boogla= [1, 2, [4, 5, 6, 7], 2390023, [3.1415926, "hoogla!"], 99]
\end{lstlisting}

can be visualized in a structure like this:

\begin{lstlisting}
1
2
4          5          6         7
3.1415926  "hoogla!"
99
\end{lstlisting}

To initialise a map in that way, it does \emph{not} need to be defined
with \Co{dim} beforehand -- actually, a square bracket initialisation
will delete any variable with the same name that may have been existing
before, and create a completely new one.

Furthermore, the \Co{array} keyword can be used as an alternative way to
initialize a map (though not an array, confusingly). Basically, what
you do is that you pass your map members in the shape of a
JSON-formatted string\footnote{>>JavaScript Object Notation<<, see eg
\href{http://en.wikipedia.org/wiki/JSON}{Wikipedia} for details}
\index{JSON} \index{JavaScript Object Notation} to the
\Co{array} function:

\begin{lstlisting}
boogla= array("[1, 2, [4, 5, 6, 7], [3.1415926, \"hoogla!\"], 99]")
\end{lstlisting}

is equivalent to the map definition above. Be aware that you pass
\emph{all} map members to \Co{array} wrapped in a single string, rather
than as individual elements. Likewise, if you use string variables in
your map, these must be escaped with backslashes inside the \Co{array}
argument (ie, \Co{\textbackslash "hoogla\textbackslash "} rather than
simply \Co{"hoogla"}.

A further neat feature is that maps of any complexity can be serialized
with a simple \Co{print} command to a file. \Co{Print} will not only
display a single member, but will format the output as a JSON string. This
means --~without going into too many details regarding file handling
here~-- that writing a complete map to a file and loading it again at a
later stage are simple three-liner tasks.

\textbf{Access to map members} \index{map member access} is possible
through two different notations. The first one contains the key to the
map member in parentheses after the map name: \index{parentheses
notation (maps and arrays)}

\begin{lstlisting}
boogla(3.1415926)= ...
\end{lstlisting}

which is similar to an array access, with the exception that the key may
be a real value, or even a string:

\begin{lstlisting}
boogla("nuffda")= ...
\end{lstlisting}

The second option uses the >>dot notation<< \index{dot notation (maps)}
familiar from other languages like >>C<<, seperating the name from the
member key with a dot \Co{.}:

\begin{lstlisting}
boogla.nuffda= ...
\end{lstlisting}

The two notations are for the most part equivalent, while the second
alternative makes sure that you're accessing a map, not an array.

For more complex maps with nested structures, you simply >>chain<< your
notations to access the lower-level members:

\begin{lstlisting}
boogla.nuffda.oingaboinga= ...
\end{lstlisting}

The notations can be mixed within a single variable access: 

\begin{lstlisting}
dim boogla
z.tschaka= "Hello, world!"
boogla("gloegk")= z
? boogla.gloegk("tschaka")

> Hello, world!
\end{lstlisting}

But this is not recommended, because it is too easy to inadvertently
mess up your dots and parentheses and accessing a non-existent map
member.\footnote{Since it isn't necessary to declare map members before
accessing them, you will not get a warning in such a case and may spend
many hours debugging.}

An interesting feature is that with the parentheses notation, a variable
may be used as key, which isn't possible with dot notation:

\begin{lstlisting}
z.tschaka= "Hello, world!"
target= "tschaka"
? z(target), z.target

> Hello, world!	0
\end{lstlisting}

When accessing \Co{z} in the third line, in the first term
\Co{z(target)}, \Co{target} is replaced with its value \Co{tschaka} by
the interpreter, before looking the map member with that name up and
returning the result, \Co{"Hello, world!"}. In the second term
\Co{z.target}, the interpreter looks for a member of \Co{z} with the key
\Co{target}, can't find one, and tacitly creates a new one which is
initialized with the value~$0$.\footnote{Wrap your heads around this,
folks. It's important.}

This makes the parentheses notation particularly useful when you want to
decide at runtime which map member you want to access: With dot
notation, access is fixed, but with parentheses notation you can pass a
variable to determine which member to use in that instance.

\subsection{\Co{len} \label{cLen}}

For both arrays and maps, \Co{len(x)} will give you the number of
elements in the structure. But note that this isn't the total number of
elements, but only the number of elements in the >>top dimension<<.

For example, when initializing an array as

\begin{lstlisting}
boogla= [1, 2, [4, 5, 6, 7], 2390023, [3.1415926, "hoogla!"], 99]
\end

\Co{len(boogla)} will be $6$, as this is the number of first-dimension
members, not 10, which would be the total number of elements.

\subsection{Almost, but Not Quite the Same (Not a Rose by any other
Name)}

The \textbf{near-but-not-quite equivalency of maps and arrays} leads to
interesting consequences, read: causes for unexpected trouble. For
example, in the following code

\begin{lstlisting}
dim x
dim z
x("hello")= 10
x(5)= 55
z(5)= 10
z("hello")= 55
\end{lstlisting}

the third and fourth line cause no errors, but the fifth one does. Why
is this? The cause is, that after the \Co{dim} command, \Co{x} and
\Co{z} essentially are still shapeless somethings, but the first access
with a string for an index forces \Co{x} to become a map (line 3).
Accessing it \emph{afterwards} with a numerical key is no problem at
all. \Co{z} on the other hand is firstly accessed as an array (line 5),
and this leads to problems, because neither has its size been defined,
nor has the appendix operator been used.

Likewise,

\begin{lstlisting}
dim x

z= "hello"
x(z)= 3.14
x << 99
\end{lstlisting}

throws an error in the last line, namely >>index out of range<<. The
first use of \Co{x} forced it to become a map, but unfortunately, the
appendix operator is only defined for arrays,\footnote{and it \emph{can}
only be defined for arrays, because what would be the key for the value
appended, if the recipient was a map?} hence the error.

Finally,

\begin{lstlisting}
zoogla= [[1, 2, 3], [10, 11, 12], [99, 98, 97]]
\end{lstlisting}

looks tantalizingly like a definition for a two-dimensional array, but
it actually is a map -- which you find out when you try to access it
like an array; \Co{zoogla(2, 2)} will cause an >>out of range<< error.
Only \Co{zoogla(2)(2)} will work.

\absatz

\quick{In a nutshell: Try to avoid arrays whenever possible. In the long
run, you're better off if you force your structures to be maps and
treat them accordingly.}

\section{Pointers and References \label{referenceOperator}}

\subsection{Referencing Variables}

\SB\ also provide a reference operator. If you're familiar with >>C<<,
which is explicitly built around the notion of pointers and references,
this concept will be nothing new for you. The essence is that the
reference operator \Co{\at}, applied to any variable, will not return the
variable's \emph{value}, but its \emph{address}, ie its memory location.
\index{\at\ (reference operator)} \index{object reference} In other words,
after

\begin{lstlisting}
y= @x
\end{lstlisting}

\Co{y} and \Co{x} will point to the same variable, and changes made to
\Co{x} will always be reflected by \Co{y}:

\begin{lstlisting}
y= @x
x= 10
? y
x= 20
? x, y

> 10
> 20		20
\end{lstlisting}

Unfortunately, this is a somewhat \footnote{We've all been
there~\ldots}, because the reverse isn't true. Continue the above code:

\begin{lstlisting}
...
y= 50
? x, y

> 20		50
\end{lstlisting}

What gives? When executing \Co{y= 50}, the value $50$ is assigned to
the variable \Co{y}, which from this point on no longer holds \Co{x}'s
address; the link between the two variables is broken, and the variables
go their seperate ways. So, as long as \Co{y} is a reference to \Co{x},
it will always reflect the current value of \Co{x}, but the opposite
isn't true: Once \Co{y}'s value changes, it is a reference to \Co{x} no
more. So, when applied \textbf{to simple variables, the reference
operator works >>read-only<<.}\footnote{and is in the words of its
inventor, >>a bit useless<<}

The situation changes when \Co{\at} is applied to a map, and a member of
that map is overwritten:

\begin{lstlisting}
dim boogla
boogla("honk")= 20
y= @boogla
y.honk= 99
? boogla.honk

> 99
\end{lstlisting}

performs as expected, and \Co{y} is now a >>true<< copy of \Co{boogla}.
The reason is that the line \Co{y.honk= 99} doesn't assign a new value
to \Co{y}, but to a member of \Co{y}, hence \Co{y}'s >>integrity<<
remains unviolated. \textbf{With maps, the reference operator creates
two-way equivalences.} Of course, when you assign a value to \Co{y}
itself (rather than to a member of \Co{y}), you will break that
relationship like with a simple variable.

Note that the use of the reference operator here differs slightly from
that in passing parameters to subroutines (see
\Cref{passingParameters}).

\subsection{Referencing Routines}

The reference operator \Co{\at} works not only on variables, but also
on routines:

\begin{lstlisting}
boogla= @honk
zoogla= @buzz
call boogla, 1, 2, 3
call @honk, 4, 5, 6
? call(zoogla, 5)

sub honk(x, y, z)
	print x*y*z
end
func buzz(x)
	buzz= x*x
end

> 6
> 120
> 25
\end{lstlisting}

The keyword \Co{call} is used to invoke a routine. It is followed by the
reference to the routine, or a variable which holds the
reference,\footnote{If you think about it, only the latter case
is really really useful.}, and a comma-separated list of parameters.
Note that if a procedure is \Co{call}ed, the parameter list \emph{must
not} be in parentheses, while when you \Co{call} a function, it
\emph{must}.



\chapter{Control Flow}

\quick{In this section we'll describe a number of ways to establish
control flow in a \SB\ program, ie everything which keeps the program
from simply executing line after line of code. It deals with conditional
operations loops, and exceptions.}

\section{Conditionals}

\subsection{\Co{if \ldots\ then}}

The most simple case is an \Co{if \ldots\ then} construct, which should be
familiar from other BASIC dialects.

In its regular form, it looks like this:

\begin{lstlisting}
if temp>20 then
	print "It's a warm day."
endif
\end{lstlisting}

The expression following \Co{if} need not be in parentheses. The keyword
\Co{then} is optional. The keyword \Co{endif} may be replaced with
\Co{fi} (which is \Co{if} reversed \ldots).

For more complex cases, alternative branches can be explored with the
keywords \Co{elseif} and \Co{else}:

\begin{lstlisting}
if temp>30 then
	print "It's really hot."
elseif temp>20 then
	print "It's a warm day."
else
	print "It's cool."
endif
\end{lstlisting}

\Co{Elseif} can be replaced with \Co{elif}. \Co{Else} will catch all
alternatives, if none of the \Co{if} and \Co{elseif} branches are true.

Note that all the branches can be tested against arbitrary expressions;
they don't need to refer to the same variable. If you want to test a
single value against several possible outcomes, \Co{select \ldots\ case}
is probably a better option, see \Cref{selectCase}.

Several \Co{if} clauses can be nested. It's your resposibility to make
sure than they are properly closed, especially when you're using many
\Co{elseif}/\Co{else} branches.

If your >>deserts are small<<,\footnote{\emph{He either fears his fate
too much\\ Or his deserts are small\\ Who dares not put it to the
touch\\ To win, or lose, it all} -- Earl of Montrose} and you don't have
to process much code in your \Co{if} clause, then there is a single-line
variation as well:

\begin{lstlisting}
if temp>30 then ? "Hot" else ? "Moderate"
\end{lstlisting}

Note that in this case, \Co{then} is mandatory, while \Co{endif} must
not be used. You can put several colon-separated commands between
\Co{then} and \Co{else} and after \Co{else}, respectively, provided you
can fit everything into a single line of code.

\subsection{\Co{iff(\ldots)}}

As with the single-line option above, there is also an >>inline<< if
clause. >>C<< users will be reminded of the \Co{x ? y : z} syntax used
there. In \SB, it is the keyword \Co{iff}, followed by a list of three
parameters. The first is the condition, the second the result of the
clause in the case the condition is true, and the third the result
otherwise. The following two examples are equivalent:

\begin{lstlisting}
nuffda= iff(hoogla, boogla, zoogla)

if hoogla
	nuffda= boogla
else
	nuffda= zoogla
endif
\end{lstlisting}

\Co{Iff} helps you to make your code more concise, and better readable.
Since \Co{iff} is simply a function, it can also show up within more
complex expressions:

\begin{lstlisting}
honka= "Hello, " + iff(its_a_boy, "dude", "chick") + "!"
\end{lstlisting}

which may or may not help with the readability of your code.

\subsection{\Co{select \ldots\ case \ldots\ end select\label{selectCase}}}

Finally, many programming languages offer a simplified syntax for
testing a single variable (or expression) against a number of
conditions, and \SB\ is no exception.

Here, such a clause is introduced with the two keywords \Co{select
case}, followed by a variable or expression. Then, a number of
conditions will be tested with \Co{case} statements, before the whole
clause is closed with \Co{end select}:

\begin{lstlisting}
nuffda= 10

select case nuffda
  case 1
    ? "1"
  case 10
    ? "10"
end select
\end{lstlisting}

Each \Co{case} is followed by an expression (variable or function)
against which the \Co{select case} expression is tested. The \Co{select
case} expression is evaluated only once, namely when entering the whole
construct.

Note that, compared to other programming languages, there are several
limitations to the construct:

\begin{itemize}

\item No \Co{break} required or even allowed. This makes it impossible
to achieve a >>fallthrough<< of several \Co{case} clauses (intentionally
or accidentally).

\item There is no way to compare for inequality (like \Co{case > 5} --
this would be an illegal construct), and 

\item There is no \Co{default} clause which would serve to catch the
cases not dealt with explicitly (analogous to the \Co{else} clause in
\Co{if} constructs).

\end{itemize}

\section{Loops}

\subsection{\Co{for \ldots\ next}}

\Co{For} loops come in two flavours with \SB:

The first is the regular loop which you are probably familiar with from
other programming languages:

\begin{lstlisting}
for i=start to end [step inc]
	...
next
\end{lstlisting}

The \Co{step} keyword and the subsequent increment \Co{inc} (which can
be any expression, not necessarily only a variable) are optional; if they're
missing, the increment is set to $1$. There is no need to add to the
\Co{next} keyword the name of the loop variable.\footnote{Currently, you
can write \emph{anything} you want there without causing an error, but I
guess this is more a bug than a feature, and will be removed over the
next few versions.}

The index loop variable \Co{i} will be set to the initial value
\Co{start}, and the code inside the loop executed at least once. Upon
reaching the corresponding \Co{next} statement, the index is compared to
the limit \Co{end} given after the \Co{to} keyword. If the index is
smaller or equal to \Co{end}, the index is incremented by the \Co{inc},
if this is provided, or by $1$), and the loop is traversed once more.
(If \Co{inc} is negative, the situation is obviously reversed.)

This means that to traverse through a complete array (assuming it uses
sequential indices only), you must configure your loop like this:

\begin{lstlisting}
dim x(423)
...
for i=0 to 423
	...
next
\end{lstlisting}

The index is considered a regular variable inside the loop, and open to
manipulation. This means that you can play tricks like:

\begin{lstlisting}
for i=0 to 100000
	...
	if i=10
		i= 1000001
	endif
next
\end{lstlisting}

Since the \Co{inc} expression is evaluated after each loop traverse, you can
mess with that as well.

The second >>flavour<< of \Co{for} is meant to deal with more complex
arrays and maps. It has a slightly different syntax:

\begin{lstlisting}
for i in z
	...
next
\end{lstlisting}

where \Co{z} is an array or map. The \Co{for} loop will be traversed
once for each member of the structure's top dimension (as evaluated by \Co{len}, see
\Cref{cLen}). The value of \Co{i} is set to:

\begin{itemize}

\item \Co{z(i)}, if \Co{i} is an array, or

\item the >>next<< key of \Co{z}, if it is a map.

\end{itemize}

In the case of a map, the map element can be accessed with
\Co{z(i)}.\footnote{Yes, that's correct. Read it again.}

\begin{lstlisting}
dim zoogla(5)
zoogla(3)= "uffda"

boogla= [[4, 5, 6, 7], 2390023, [3.1415926, "hoogla!"], 99]
boogla("tchaka")= 500
boogla.bonka= 999

for x in zoogla
  ? x
next
?
for x in boogla
  ? x, boogla(x)
next 

> 0
> 0
> 0
> uffda
> 0
> 0
>
> 0	[4,5,6,7]
> 1	2390023
> 2	[3.1415926,hoogla!]
> 3	99
> BONKA	999
> tchaka	500
\end{lstlisting}

Since it's only determined at runtime which keys are used to point to
map members, this method is necessary to make it possible to traverse
through all map members in a loop.

For maps, there is no defined order in which the keys will be allotted
to the index variable.

\subsection{\Co{while \ldots\ wend} and \Co{repeat \ldots\ until}}

When the number of times a loop is supposed to be executed is not known
beforehand (for example, when reading lines from a file when the file
length is unknown), \SB\ offers two different loop constructs:

\begin{lstlisting}
while (expression)
	...
wend

repeat
	...
until (expression)
\end{lstlisting}

In both cases the code block between the loop delimiters will be
repeated until an expression will be fulfilled. Note two important
differences though:

\begin{itemize}

\item In a \Co{while \ldots\ wend} loop, the loop is executed as long as
the expression is \emph{true} (ie, unequal to $0$), whereas a \Co{repeat
\ldots\ until} loop is executed as long as the expression is
\emph{false}, or $0$.

\item In a \Co{while \ldots\ wend} loop, the test for the expression is
performed at the \emph{beginning} of the loop, but in a \Co{repeat \ldots\
until} loop, the expression test takes place at the \emph{end} of the loop.
This has the consequence that the \Co{repeat \ldots} code block is
guaranteed to be executed at least once, wheres the \Co{while \ldots}
code block is not.

\end{itemize}

\Co{(expression)} can be any valid term which will result in a value
returned, like a variable or a function call. It can even be useful to employ
a constant here, namely when you want to break from the loop somewhere
in the middle of the code block. For example --

\begin{lstlisting}
while 1
	' read user input
	...
	if user_name=correct
		? "Name ", user_name, " is correct."
		exit
	endif
	? "Illegal input"
wend
\end{lstlisting}

In this case your loop should contain an \Co{exit} statement (see below) to
break out of the loop.

This also serves to emulate a \Co{do \ldots\ loop} \index{do-loop}
construct that would allow for a loop to be executed >>indefinitely<<
which \SB\ doesn't feature genuinely.

\subsubsection{Pathological Cases}

It's syntactically legal to omit the expressions for \Co{while} or
\Co{until} completely. In this case the >>expression<< is always taken
to evaluate to \Co{$0$}.

With a \Co{while \ldots wend} loop this doesn't really make sense; the
code inside the loop will simply never be executed. In a \Co{repeat
\ldots until} loop though the situation is different: This loop will
endlessly be executed, and in effect such a construct without an
expression for \Co{repeat} will be equivalent to \Co{do \ldots loop}
constructs of other languages.

If you employ such a scheme, make sure that you provide a way to leave
the loop, like for example an \Co{exit} statement:

\subsection{\Co{exit}}

The keyword \Co{exit} lets you exit immediately from the innermost loop
it is found in. (This is equivalent to the >>C<< statement \Co{break}.)
You can specify a qualifier with \Co{exit}, namely one of \Co{for},
\Co{loop}, \Co{sub}, or \Co{func}, which will make \SB\ leave the
innermost surrounding structure of that type. (\Co{loop} includes
\Co{repeat} and \Co{while} constructs.)

\section{Exceptions: \Co{try \ldots\ catch \ldots\ throw}}

Exceptions provide a fairly comfortable way to catch runtime errors
\index{runtime error, catching} occuring unexpectedly in your program.
Of course, they can't help with faulty program logic. Rather, exceptions
are supposed to handle files not conforming to an expected format,
hardware problems and the like.

Formally, an exception block somewhat resembles a \Co{select \ldots\ case}
sequence. It consists of an outer >>bracket<< of \Co{try} and \Co{end
try} keywords, which delimites the >>regime<< of code to which the
exception handling applies.\footnote{Obviously, in different sections of
your code you may want to respond to the same error in different ways,
thus there's no >>global<< treatment.}
Inside this bracket there are one or more \Co{catch} sections, each of
which applies to one particular error condition:

\begin{lstlisting}
try
	' error-generating section
	...
	catch error1
		' dealing with the first error case
		...
	catch error2
		' ... second error case
		...
	' and so on
end try
\end{lstlisting}

You have basically two options to catch errors this way:

Firstly, as shown above, you may provide \emph{several}
\Co{catch}-phrases.\footnote{if you'll pardon the pun} In this case,
\Co{error1}, \Co{error2} and so on must be string expressions. Once an
error is raised, these string expressions are compared to the error
message associated with the error, and the first \Co{catch} section
which matches the error message will be executed,\footnote{Which means
that, to use the exception mechanism responsibly, you must have a good
idea what the error messages you may encounter will look like.} whereupon
the \Co{try \ldots\ catch} section will be left and the >>regular<<
surrounding code will be resumed. If none of the \Co{catch} expressions
matches, program execution is resumed after \Co{end try}, too.

Your second option is to provide only a \emph{single} \Co{catch}. In
this case, \Co{error1} must be a simple string variable, and the current
error string will be assigned to this variable (provided any error
occured at all). The corresponding \Co{catch} section will then be
executed, regardless of the exact nature of the error.

The second option is thus preferrable if you either want to have a
simple >>catch all<< which will deal with any imaginable error in a
single sweep, or, at the other extreme, if the error conditions you
expect to encounter are so confusing that you'd rather dedicate some
more sophisticated code to them than simple string comparisons against
the error messages.

If no error is caused in the error-generating section, then none of the
\Co{catch} sections are executed. Errors raised outside the \Co{try
\ldots\ catch} section can't be evaluated inside it. (The will have
caused your program to halt already.)

If no error
occured, but you feel facetious, you can use \textbf{\Co{throw} to
create any desired error}. The syntax is simply -- 

\begin{lstlisting}
throw my_err
\end{lstlisting}

with the parameter \Co{my\_err} being the error string. (Outside of a
\Co{catch \ldots\ try} block, \Co{throw} will cause the program to
abort.)


\chapter{Structuring a Program}

\quick{This chapter will give you an overview about how you can avoid
producing the notorious >>spaghetti code<<, and structure your program
instead into blocks which are easier to debug and maintain.}

\section{Routines: Procedures and Functions}

Routines (also called >>subroutines<<) are blocks of code set apart from
the main code. This can be done for a variety of reasons, for example
simply to break down a complex task into individual stages which are
more readily analyzed and maintained. Another reason is reusability; if
the program needs to perform the same task in several stages, it's more
economical to write the code once and reuse it as is necessary.

Routines come in two flavours: >>Procedures<< and >>functions<<.
Syntactically, in \SB\ procedures and functions are almost equivalent.
The only difference is that a function returns a value when called,
whereas procedures do not. \textbf{A function can only return a single
variable}, but this may be an arbitrarily complex map. If you need to
manipulate more than a single value, you can also pass parameters by
reference, see \Cref{passingParameters} below.

\subsection{Definition}

Procedures and functions are defined by embracing a block of code
between the \Co{sub} or \Co{func} keyword, resp., at the beginning
followed by the routine's name, and \Co{end} at the end of the block.
Parameters are defined as a comma-separated list of variables following
the routine name in parentheses:

\begin{lstlisting}
sub x(hoogla, boogla)
	...
end

func y(arg1, arg2, arg3)
	...
	y= arg1+arg2
	...
end
\end{lstlisting}

For a function, the \textbf{return value} is determined by assigning an
expression to a variable with the same name as the function, in the
example above in the line \Co{y= arg1+arg2}.

Note that this is in contrast with most other BASIC dialects which use
the keyword \Co{return} instead. There, \Co{return} also makes the
interpreter exit the routine and return control to the calling code immediately.
Not so in \SB: Here, \textbf{\emph{all code} up to the \Co{end} keyword
is executed}, with all side effects it may generate.

Routines may be \textbf{defined anywhere} in your code; they don't need to be
defined before they are invoked.

\subsection{Arguments}

Arguments\footnote{Most times a disctinction between >>arguments<< and
>>parameters<< is made in computer literature, but we'll treat both as
synonyms.} are passed to a procedure or a function when invoking it as a
list of comma-separated variables and constants following the routine
name. When invoking a procedure, the parentheses are optional, note
though that you use this feature at your own risk. \textbf{Parameter
lists in the definition and the call must match} in the number of
arguments.

\begin{lstlisting}
y(10, 20)
...
sub y(arg1, arg2, arg3)
	...
end
\end{lstlisting}

is not legal.

Calling a function and \textbf{not using the return value} is no problem:

\begin{lstlisting}
y(10, 20)
...
func y(arg1, arg2)
	y= arg1*arg2
end
\end{lstlisting}

will cause no error. The return value is simply discarded. In contrast,
calling a procedure \emph{in lieu} of a function will confuse the
interpreter:

\begin{lstlisting}
my_result= x(10, 20)
...
sub x(hoogla, boogla)
	...
end
\end{lstlisting}

creates an error.

\subsection{Variable Scope}

Routines help with the modularization of code by >>encapsulating<< the
data, which means that routines have only access to a sub-set of all
variables defined in the program. Most importantly, routines can't read
or write variables defined in other routines. Hence it's impossible that
they would accidentally overwrite other variables. Likewise the routines
also maintain their own >>household<< of variables accessible only to
them.

The keyword \Co{local} is used to define variables >>attached<< to
a routine. \index{local variables} The variables come into existance the
minute the routine is invoked, and they're deleted again as soon as the
routine is terminated. If a local variable (or a routine parameter) has
the same name as variable previously defined (in the main program or a
routine which called the current routine), the previous instance is
>>shadowed<<, and the routine will access the local variable instead,
until the current routine is left again. A local variable will also be
visible to a routine which is called from the routine where the local
was defined.

The following code may explain the behaviour. It differs in important
details from that of other programming languages and BASIC dialects:

\begin{lstlisting}
nagaqk= 100
zoogla= 200
gluck
? "In main:", nagaqk, zoogla

sub gluck
  local nagaqk
  nagaqk= 30
  zoogla= 200
  boogla
  ? "In gluck:", nagaqk, zoogla
end

sub boogla
  ? "In boogla:", nagaqk, zoogla
  nagaqk= 15
  zoogla= 99
end

> In boogla:	30		200
> In gluck:		15		99
> In main:		100	99
\end{lstlisting}

Let's have a look at what is actually happening here. First, the global
variables \Co{nagaqk} and \Co{zoogla} are defined and assigned the
values $100$ and $200$, resp. Next, \Co{gluck} is invoked and defines a
local variable \Co{nagaqk} which >>shadows<< the global variable of the
same name. Thus, the value $30$ is assigned to the local instance of
\Co{nagaqk}, not to the global one. As opposed to that, there only is
one instance of \Co{zoogla}, and the value $200$ is written to that.

Next, \Co{boogla} is called, which has access to all the >>knowledge<<
\Co{gluck} has. When the old values of \Co{nagaqk} and \Co{zoogla} are
overwritten, this happens again to the local copy of \Co{nagaqk}, but to
the global instance of \Co{zoogla}. Had \Co{boogla} defined its own
local copy of \Co{nagaqk}, \emph{that} copy would have been overwritten
rather than \Co{gluck}'s.

The writing done in \Co{boogla} is still >>felt<< in \Co{gluck} when
control returns there. But when \Co{boogla} is left, its local instance
of \Co{nagaqk} is deleted and the original instance (defined globally)
returns to the surface unscathed. Note that for \Co{zoogla} there only
ever was a single instance. Had \Co{boogla} had its own instance of
\Co{nagaqk}, the results would also have been different.

Note that local variables can be defined anywhere in the routine. But if
you access a variable before it's defined as local, you will actually
create a new \emph{global} variable first:

\begin{lstlisting}
sub hoogla
	zoot= 100

	local zoot
	zoot= 10
	? zoot
end hoogla

> 10
\end{lstlisting}

This creates (or overwrites) a global variable with the name \Co{zoot}
and the value $100$, then creates a local variable with the same name,
assigns it the value $10$, and then destroys the local copy at the end
of the procedure, while the global copy still lives on.

Routines can \textbf{recurse}, ie invoke themselves again before they're
finished.\footnote{At least, they can do so to a reasonable degree of
levels.} \index{recursion} Every time a new instance of the routine is
called, it will also create a new set of parameters and local variables,
while the old set is >>put aside<< and only restored when the execution
of the current routine level is finished.

\begin{lstlisting}
hoogla

sub hoogla
  local zoogla
  
  level= level+1
  zoogla= level
  if level<5 then
    hoogla
  endif
  ? zoogla
end

> 5
> 4
> 3
> 2
> 1
\end{lstlisting}

The definitions of \textbf{routines may be >>nested<<}, \index{nested
routines} ie one routine (the >>child<<) may be defined within the code
block of another (the >>parent<<).\footnote{Don't confuse the
terminology here with child and parent processes/threads.} Whether you
define a routine inside or outside another routine has little bearing on
the variables household of the child routine. But the child routine is
only visible from inside the parent routine and its >>siblings<<. To any
code outside the parent routine, the child will be invisible:

\begin{lstlisting}
hoogla

child1

sub hoogla
  local zoogla

  child2
  sub child1
    ? "Here I am"
  end
  
  sub child2
   child1
	? "I'm here, too"
  end
end
\end{lstlisting}

causes an error in the third line, because \Co{child1} is invisible
outside \Co{hoogla}. The rest of the code will be executed fine if you
comment out the third line.

\SB\ provides nothing in the way of \textbf{static variables}, ie local
routine variables which maintain their value between two subsequent
calls of the routine. \index{static variables (non-existant)}

\subsection{Passing Parameters \label{passingParameters}}

Per default, parameters are passed to procedures and functions
\textbf{>>by value<<}, which means that copies of the arguments are
created for the routine. \index{by value (parameter passing)} Changing
these copies will have no effect on the variable in the calling code;
both instances are independent of each other. This is true \textbf{even
for maps and arrays}. This behaviour comes with a certain penalty,
namely when you work with complex maps and do a lot of recursion. In
this case, the interpreter is busy with copying lots of data which will
also require a lot of memory.

To avoid this, you can require in the definition of a routine that some
parameters will be passed >>by reference<<. \index{by reference
(parameter passing)} In this case, no local copy will be created, but
the routine will work on the same data as the calling code does: Changes
to the value of a parameter are propagated to the caller. To employ
passing by reference, the respective parameters in the routine
definition must each be preceded with the keyword \Co{byref}, or the
reference operator \Co{\at}: \index{\at\ (reference operator)}
\index{object reference}

\begin{lstlisting}
bunga= 10
chaka= 20

hoogla(bunga, chaka)
? bunga, chaka

sub hoogla(zoogla, byref boing)
  zoogla= 99
  boing= 101
end

> 10, 101
\end{lstlisting}

Besides reducing CPU power and memory required, passing parameters by
reference has the additional effect that a routine can write on the
parameters passed. This enables you to write procedures which change
more than one global variable at a time. Bear in mind that the
\emph{calling code} has no way to >>see<< whether it passes a variable
by value or by reference; the behaviour is completely in the hand of the
\emph{called routine}.

Notice that this behaviour is subtly different from the use of the
reference operator \Co{\at} with a regular variable, see
\Cref{referenceOperator}. You can (for obvious reasons) not apply the
reference operator inside the routine's code to a parameter or a local
variable.

\subsection{One-line Functions}

Sometime the code required for a function is short and neatly fits into
one line. In this case, \SB\ provides a more concise syntax for function
definitions, namely with the keyword \Co{def}:

\begin{lstlisting}
def hoogla(x)= sin(x)*cos(x)

func zoogla(x)
	zoogla= sin(x)*cos(x)
end
\end{lstlisting}

Both definitions above for \Co{hoogla} and \Co{zoogla} are equivalent.

This does not work for procedures.

\section{Modules \label{modules}}

To modularize your code above the level of routines, \SB\ offers the
option to include other source files, and to create libraries of
>>units<<.

\subsection{File Inclusion}

In its most simple form, \SB\ lets you import other source files into
the current file at runtime:

\begin{lstlisting}
include "bunga.bas"
\end{lstlisting}

in the code will make the contents of the file \Co{bunga.bas} available
to the file currently running in the interpreter. >>First level<< code
\footnote{ie, code outside any routines} in \Co{bunga.bas} (will be
executed immediately.\footnote{It is an interesting experiment to create
such an \Co{include}-file during program runtime and import it then.
Effectively, such a program would >>bootstrap<< itself. Not for the
faint at heart.} If the included file contains a routine with the same
name as one defined in the >>mother<< file, an error occurs; the old
version of the routine is \emph{not} replaced.

Think of it as a simple copy-paste operation.

\subsection{Units}

>>Units<< are a more sophisticated concept in \SB\ which allows the creation
of genuine program libraries with their own namespace and well-defined
interfaces.

Units are kept in seperate source files; each file contains exactly one
unit which bears the same name as the file \emph{sans} the \Co{.bas}
extension.\footnote{I was informed that this isn't \emph{strictly} true,
but you can cause great confusion in the IDE if you don't stick to that
convention.}

\begin{lstlisting}
file hoogla.bas:
...
unit hoogla
...
export zoogla, boogla
zoogla= "Hello world!"

sub boogla(name)
	print "Goodybe", name, "!"
end
\end{lstlisting}

Inside the unit file, you can write code as you would in any source
file, and define variables (simple and composite) and routines
(procedures and functions). All of these variables and routines are
local to the unit file, unless they're defined to be public with the
keyword \Co{export}.

First level code is executed when the library is loaded, but it takes
place in a separate namespace, ie a variable called \Co{chaka} in the
unit file will not conflict with a variable with the same name in the
mother file; they're two seperate entities.

To use a unit, it must be first be compiled into bytecode. You can do so
from the IDE, or use the command line:

\begin{lstlisting}
sbasici hoogla.bas
\end{lstlisting}

which creates a file \Co{hoogla.sbu}. This must be repeated after
updates to the unit file. Then it can be loaded with the keyword
\Co{import} in the mother file which is to use the library. From
this moment on, all \Co{export}ed variables and routines are available
to the mother file. Their name there is a combination of the unit name,
a dot \Co{.} and the variable or routine's >>proper<< name. With the
above code segment from \Co{hoogla.bas} you get:

\begin{lstlisting}
file ragaqk.bas:

import hoogla
? hoogla.zoogla
hoogla.boogla("Clint")

> Hello world!
> Goodbye,	Clint	!
\end{lstlisting}

It should be painfully obvious that a unit can't import itself again.


\chapter{User Interface with \SB \label{cUse}}

\quick{How to program output on the screen, and how to receive user
input.}

\absatz

\SB's display offers a choice of modes or >>philosophies<< to program
user interaction:

\begin{itemize}

\item The console, emulating a text terminal in the manner of a VT220

\item A number of graphic primitives which allow you to draw text,
lines, circles, etc.

\item A graphical user interface with more sophisticated items to create
dialogs and (simple) forms.

\end{itemize}

Bear in mind, though, that those three modes coexist. There is no switch
from one mode to the other (as it used to be in the old days, when one
switched from text to graphics mode), but you can freely mix commands
from between the three modes (and thereby make a mess from your screen).

\textbf{Colors} can be passed as parameters to various commands in two
different ways:

\begin{itemize}

\item As a simple number between $0$ and $15$, in which case it is meant
to represent a standard DOS color.

\item As RGB value.\footnote{Theoretically this is limited to devices
with a sufficiently capable graphics system, but I doubt there are many
systems in existance which wouldn't fullfil that requirement.}

\end{itemize}

To pass an RGB value, simply call the \Co{RGB()} or \Co{RGBF()}
function with three parameters as the Red -- Green -- Blue values. In
the case of \Co{RGB} the parameters must be between $0$ and $255$, in the
case of \Co{RGBF} (>>float<<) between $0$ and $1$.

\section{The Console \label{cConsole}}



\section{Graphic Primitives \label{cGraphics}}

\section{Graphical User Interface \label{cGUI}}


\part{Reference}

\chapter{Specials}

\quick{This chapter provides you with an overview over special characters
and pre-defined constants in \SB.}

\section{Characters and Operators}

\absatz

\begin{tabular}{|c|l|l|}
\hline
$<<$ & array appendix & \\
\hline
' & comment & \\
\hline
\# & comment, >>shebang<< & \\
\hline
\& & line continuation & see \ldots \\
\hline
\&. & binary/octal/hex notation & \\
\hline
$0.$ & binary/octal/hex notation & \\
\hline
" & string delimiter & \\
\hline
+ & plus sign, string concatenation & \\
\hline
: & command separator & \\
\hline
= & assignment and comparison & \\
\hline
$[\ldots]$ & map initializer & \\
\hline
\$ & not a string sigil & \\
\hline
\at & variable/routine reference & \\
\hline
\end{tabular}

\section{Pre-defined Constants}

These constants are pre-defined through the interpreter. They're
avaiable inside any \SB\ program, and their values can't be changed.
Consequently, you also can't define your own variables with the same
names as any of these constants.

If you're missing some required values here, check the reference
section: \Cref{keywordReference}. Some constants are actually system
functions.

\absatz

\begin{tabular}{|l|l|}

\hline
TRUE & $1$, in other contexts any value different from $0$ \\
\hline
FALSE & $0$ \\
\hline
OSNAME & Operating System name\\
\hline
OSVER & Operating System Version\\
\hline
SBVER & \SB\ Version \\
\hline
PI & The mathematical constant, $3.1415926...$ \\
\hline
XMAX & Graphics: Width of the window in pixels$\dagger$ \\
\hline
YMAX & Graphics: Height of the window in pixels$\dagger$ \\
\hline
BPP & Graphics: bits per pixel (color resolution) \\
\hline
VIDADR & Video RAM address (only on specific drivers)\\
\hline
CWD & Current working directory\\
\hline
HOME & User's home directory\\
\hline
COMMAND & Command-line parameters passed to the program\\
\hline
\end{tabular}

\absatz

$\dagger$) This value is \emph{not} updated if the dimensions of the
window are changed during the run of the program.

\section{Supported ESC-Codes in Console Mode}

For text-only terminals, a set of
\href{http://en.wikipedia.org/wiki/ANSI_escape_code}{ANSI escape
codes}\footnote{Check out the Wiki article and see which codes you can
get to work on your machine. >>You can't permanently damage your
computer that way<<, as the manual for my old Commodore~64 used to say.}
has been defined to set various pseudo-graphics attributes. \SB\
supports a partial set of these graphics and adds a few extensions of
its own to it.

Most of the commands begin with the >>Command Sequence Initializer<<, or
CSI for short. This is a sequence of two characters with ASCII codes 27
and 91, which are the \Co{escape} control code and an opening square
bracket \Co{[}, resp. In the following table, this sequence is represented
as \Co{\textbackslash e[}. For the other control sequences, the first
letter is simply a backslash. \absatz

\begin{tabular}{|l|l|}
\hline
\textbackslash t & tab (20 px) \\
\hline
\textbackslash a  & beep \\
\hline
\textbackslash r  & return \\
\hline
\textbackslash n  & next line \\
\hline
\textbackslash xC  & clear screen (new page) \\
 & \\
\hline
\textbackslash e[K   & clear to end of line \\
\hline
\textbackslash e[0m  & reset all attributes to their defaults \\
\hline
\textbackslash e[1m  & bold on \\
\hline
\textbackslash e[4m  & underline on \\
\hline
\textbackslash e[7m  & reverse video \\
\hline
\textbackslash e[21m  & bold off \\
\hline
\textbackslash e[24m  & underline off \\
\hline
\textbackslash e[27m  & reverse off \\
\hline
\textbackslash e[3nm  & foreground color, see below\\
\hline
\textbackslash e[4nm  & background color, see below\\
& \\
\hline
\textbackslash e[ A  & Displays an alert box \\
\hline
\textbackslash e[ K  & Displays the virtual keyboard \\
\hline
\textbackslash e[ L  & Displays a label at the bottom of the screen \\
\hline

\end{tabular}

\absatz

The set of colours supported is:\\
0 black, 1 red, 2 green, 3 brown, 4 blue, 5 magenta, 6 cyan, 7 white\\
So, \Co{\textbackslash e[42m} would set your background colour to green.


\chapter{Keywords and Built-in Functions \label{keywordReference}}

% here come various phrases which repeat throughout the reference
% section:

\newcommand{\eArgNumber}{The argument must be a number.}

% here are the actual command references:

\newcommand{\rAbs}{\vCommand{abs}
{Return the absolute value of the argument}
{x= abs(y)}
{Serves to determine the absolute value of its argument \Co{y}. \eArgNumber}
\vCommend{\sbref{sgn}}}

\newcommand{\rSgn}{\vCommand{sgn}
{Returns the sign of its argument}
{x= sgn(y)}
{Return value is
\[\text{\co{sgn(y)}}= 
\begin{cases} -1 & \text{if } y<0 \\
0 & \text{if } y=0 \\
1 & \text{if } y>0
\end{cases}\] \eArgNumber}
\vCommend{\Cref{noteOnTypes}, \sbref{abs}}}




\section{Alphabetical List}

Todo:

? ! ~ ABS ABSMAX ABSMIN ACCESS ACCESS ACOS ACOSH ACOT ACOTH ACSC ACSCH
AND APPEND ARC ASC ASEC ASECH ASIN ASINH AT ATAN ATAN2 ATANH ATN BALLOC
BAND BCOPY BCS BEEP BGETC BIN BLOAD BOR BPUTC BSAVE BUTTON BYREF CALL
CASE CAT CBS CDBL CEIL CHAIN CHART CHDIR CHMOD CHOP CHR CINT CIRCLE
CLOSE CLS COLOR CONST COPY COS COSH COT COTH CREAL CSC CSCH DATA DATE
DATEDMY DATEFMT DEF DEFINEKEY DEG DELAY DELETE DERIV DETERM DIFFEQN DIM
DIRWALK DISCLOSE DO DOFORM DRAW DRAWPOLY ELIF ELSE ELSEIF EMPTY ENCLOSE
END ENDIF ENV ENV ENVIRON ENVIRON EOF ERASE EQV EXEC EXIST EXIT EXP
EXPORT EXPRSEQ FI FILES FIX FLOOR FOR FORMAT FRAC FRE FREEFILE FUNC
GOSUB GOTO HEX HTML IF IFF IMAGE IMAGEH IMAGEW IMP IMPORT IN INKEY INPUT
INPUT INPUT INSERT INSTR INT INTERSECT INVERSE ISARRAY ISDIR ISFILE
ISLINK ISNUMBER ISSTRING JOIN JULIAN KILL LABEL LINEINPUT LINPUT LBOUND
LCASE LEFT LEFTOF LEFTOFLAST LEN LET LIKE LINE LINEQN LOCAL LOCATE LOCK
LOF LOG LOG10 LOGPRINT LOWER LTRIM M3APPLY M3IDENT M3ROTATE M3SCALE
M3TRANS MAX MALLOC MDL MID MIN MKDIR MOD NAND NEXT NOR NOSOUND NOT OCT
ON OR OPEN OPTION PAINT PAUSE PEEK PEEK16 PEEK32 PEN PEN PLAY PLOT POINT
POLYAREA POLYCENT POLYEXT POW PRINT PROGLINE PSET PTDISTLN PTDISTSEG
PTSIGN RAD RANDOMIZE READ RECT REDIM REM RENAME REPEAT REPLACE RESTORE
RETURN RGB RGBF RIGHT RIGHTOF RIGHTOFLAST RINSTR RMDIR RND ROOT ROUND
RTE RTRIM RUN SEARCH SEC SECH SEEK SEEK SEGCOS SEGLEN SEGSIN SELECT SEQ
SGN SIN SINH SINPUT SORT SOUND SPACE SPC SPLIT SPRINT SQUEEZE SQR
STATMEAN STATMEANDEV STATSPREADP STATSPREADS STEP STKDUMP STOP STR
STRING SUM SUMSQ SWAP SUB TAB TAN TANH TEXT TEXTHEIGHT TEXTWIDTH THEN
TICKS TICKSPERSEC TIME TIMEHMS TIMER TLOAD TRANSLATE TRIM TROFF TRON
TSAVE TXTH TXTW UBOUND UCASE UNIT UNTIL UPPER USE VADR VAL VIEW WEEKDAY
WEND WHILE WINDOW WRITE XOR XPOS XNOR YPOS 

\rAbs

\rSgn

\section{Commands by Category}

\subsection{Language Keywords}

\subsection{Mathematics, and all Things Calculated}

\begin{itemize}
\item \sbref{abs}
\item \sbref{sgn}
\end{itemize}

\subsection{String Handling}

\subsection{User Interface}

\quick{For these commands in context, see also \Cref{cUse}}

\subsubsection{Console}

\subsubsection{Graphics}



\subsubsection{GUI Operation}



\subsection{File Handling}

\subsection{Interfacing with the Operating System}

\subsection{The Rest}




\backmatter

\appendix

\printindex

%**********************************************************************

\input{../LaTeX/signature}

\end{document}


