%**********************************************************************
% additional commands for LaTeX/the Yabasic Vademecum
%**********************************************************************

\newcommand{\SB}{\textsc{SmallBASIC}}

\newcommand{\SBversion}{V 0.10.16}

\newcommand{\blindtext}{Lorem ipsum dolor sit amet}

%**********************************************************************

%\newcommand{\newFeature}{\marginpar{\textsf{\small \begin{center}
%\includegraphics[width=0.8cm]{icon/new} \\ New in V3\end{center}}}}

% allows the use of funky icons on the right margin for warnings etc.

%**********************************************************************

% what does this comment do?
\newsavebox{\fmbox}
\newenvironment{fmpage}[1]
{\begin{lrbox}{\fmbox}\begin{minipage}{#1}}
{\end{minipage}\end{lrbox}\fbox{\usebox{\fmbox}}}

% give a quick summary of the chapter:

%newcommand{\quick}[1]{\begin{itemize} \item #1 \end{itemize}}

\newcommand{\quick}[1]{\fbox{\parbox{\textwidth}{#1}}}

\newcommand{\gist}[1]{\marginpar{\footnotesize #1}}

% display something as code:
\newcommand{\co}[1]{\texttt{{#1}}}

\newcommand{\Co}[1]{>>\co{#1}<<}

% references to commands start with "c:"
% references to sections start with "c"

% insert a reference to a different command in the manual section:
\newcommand{\sbref}[1]{\hyperref[c:#1]{\co{#1}}}

% insert a reference to a chapter. Argument is a label.
\newcommand{\Cref}[1]{\ref{#1}\ \nameref{#1}}

% create a complete entry in the manual section:
\newcommand{\vCommand}[4]{\subsection*{\co{#1} \label{c:#1}} {#2} \par
\medskip \textbf{Syntax:} \\ \co{#3} \par \medskip \textbf{Description:}
\\ {#4} \par \medskip}
%	* command name
%	* short description
%	* syntax
%	* long description

\newcommand{\vCommend}[1]{\par \textbf{See also: #1}}

\newcommand{\at}{\makeatletter @\makeatother}

%**********************************************************************

\newcommand{\stringCausesError}{Passing a string value or a string
expression raises a syntax error.}

\newcommand{\numericCausesError}{Passing a numerical constant, variable
or expression raises a syntax error.}

\newcommand{\twoEquiv}{The following two code segments are equivalent:}


%**********************************************************************
\newcommand{\vquote}[2]{\begin{quote} \raggedleft \emph{#1} \\ {#2}
\end{quote}}
